%-------------------------------------------------------------------------------
%	SECTION TITLE
%-------------------------------------------------------------------------------
\cvsection{近期项目  \emoji{brick}}


%-------------------------------------------------------------------------------
%	CONTENT
%-------------------------------------------------------------------------------
\begin{cventries}

%---------------------------------------------------------

  \cventry
    {上海孚盟软件有限公司} % Organization
    {Fumamx hybrid App} % Project
    {中国上海} % Location
    {2021.01} % Date(s)
    {
      \begin{cvitems} % Description(s)
        \item {技术: Vue.js; postmessage API}
        \item {将React Native App迁移到混合App,以便用户可以获得更好的体验并使用更多原生功能。}
        \item {在vue路由中添加监听原生app的监听器,并封装装了postMessage api来将消息发送到原生app端。}
        \item {管理1位iOS工程师和1位Android工程师,跟踪他们的进度并召开会议以促进更好的协作。}
      \end{cvitems}
    }

%---------------------------------------------------------

    \cventry
    {上海孚盟软件有限公司} % Organization
    {CMS静态网站} % Project
    {中国上海} % Location
    {2021.03} % Date(s)
    {
      \begin{cvitems} % Description(s)
        \item {技术: handlebar.js; tailwind.css; gulp.js; webpack.js; velocity }
        \item {通读了velocity文档,并将gulp-velocity插件导入gulp任务中,以便自动生成velocity模板文件。}
        \item {通过使用甘特图制定开发计划以与团队合作,使开发进度看起来更清晰。}
      \end{cvitems}
    }

    \cventry
    {阳狮集团} % Organization
    {IQOS EDM 邮件模板} % Project
    {中国上海} % Location
    {2020.04} % Date(s)
    {
      \begin{cvitems} % Description(s)
        \item {技术: mjml; gulp.js; Litmus }
        \item {跨3个时区团队合作,在会议上报告工作,并与后端进行实时电话讨论来更好地在crm系统上部署邮件模板。}
        \item {通过媒体查询构建响应式设计,使用gulp.js自动上传文件,使用Litums调试每个平台的邮件模板。}
        \item {加入了mjml官方Slack社群,直接询问解决方案和建议,并迅速解决自己遇到的问题。}
      \end{cvitems}
    }
    
%---------------------------------------------------------
\end{cventries}
