%-------------------------------------------------------------------------------
%	SECTION TITLE
%-------------------------------------------------------------------------------
\cvsection{工作经验 \emoji{briefcase}}

%-------------------------------------------------------------------------------
%	CONTENT
%-------------------------------------------------------------------------------
\begin{cventries}

%---------------------------------------------------------
  \cventry
    {高级前端开发工程师经理} % Job title
    {\emoji{office-building} 上海孚盟软件有限公司} % Organization
    {上海} % Location
    {2020.12 -- 至今 } % Date(s)
    {
      \begin{cvitems} % Description(s) of tasks/responsibilities
        \item{为混合应用程序创建Js Bridge使H5 和原生App之间能够通讯,从而解决了由网络和延迟引起的问题,并改善了许多用户体验,受到70\%用户的好评。}
        \item{负责并修复公司主要产品的各种问题,并使用版本控制系统(例如Git)与团队成员进行协作以组织修改和分配任务。}
        \item{管理1名工程师,1名iOS工程师和1名Android工程师;领导并设计针对每个开发阶段需求的方法。}
        \item{使用velocity模版建立CMS网站,创建gulp任务以自动编译velocity,以节省大量调试时间。}
      \end{cvitems}
    }

%---------------------------------------------------------
  \cventry
    {软件工程师} % Job title
    {\emoji{office-building} 阳狮集团-上海璞砺营销咨询有限公司} % Organization
    {上海} % Location
    {2019.04 -- 2020.08} % Date(s)
    {
      \begin{cvitems} % Description(s) of tasks/responsibilities
        \item{实现大量的Minisite的UI,例如MBBF,HDC等一系列华为活动和会议。}
        \item{为Marriott,DS auto和SAIC motor建立了富有创意和动效的官方网站。}
        \item{与团队成员跨3个时区(新加坡,日本,印度)合作,并按时提供了3周的邮件营销模版,节省了公司人力资源。}
        \item{管理了2个月软件实习生;领导了他项目的起步和框架设计;得到了领导能力和解决问题的良好口碑}
      \end{cvitems}
    }

%---------------------------------------------------------
\end{cventries}
